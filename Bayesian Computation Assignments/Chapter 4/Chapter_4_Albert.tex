%\documentclass[12pt,addpoints]{exam}   % Print w/o solutions
\documentclass[12pt,addpoints,answers]{exam}\usepackage[]{graphicx}\usepackage[]{color}
%% maxwidth is the original width if it is less than linewidth
%% otherwise use linewidth (to make sure the graphics do not exceed the margin)
\makeatletter
\def\maxwidth{ %
  \ifdim\Gin@nat@width>\linewidth
    \linewidth
  \else
    \Gin@nat@width
  \fi
}
\makeatother

\definecolor{fgcolor}{rgb}{0.345, 0.345, 0.345}
\newcommand{\hlnum}[1]{\textcolor[rgb]{0.686,0.059,0.569}{#1}}%
\newcommand{\hlstr}[1]{\textcolor[rgb]{0.192,0.494,0.8}{#1}}%
\newcommand{\hlcom}[1]{\textcolor[rgb]{0.678,0.584,0.686}{\textit{#1}}}%
\newcommand{\hlopt}[1]{\textcolor[rgb]{0,0,0}{#1}}%
\newcommand{\hlstd}[1]{\textcolor[rgb]{0.345,0.345,0.345}{#1}}%
\newcommand{\hlkwa}[1]{\textcolor[rgb]{0.161,0.373,0.58}{\textbf{#1}}}%
\newcommand{\hlkwb}[1]{\textcolor[rgb]{0.69,0.353,0.396}{#1}}%
\newcommand{\hlkwc}[1]{\textcolor[rgb]{0.333,0.667,0.333}{#1}}%
\newcommand{\hlkwd}[1]{\textcolor[rgb]{0.737,0.353,0.396}{\textbf{#1}}}%
\let\hlipl\hlkwb

\usepackage{framed}
\makeatletter
\newenvironment{kframe}{%
 \def\at@end@of@kframe{}%
 \ifinner\ifhmode%
  \def\at@end@of@kframe{\end{minipage}}%
  \begin{minipage}{\columnwidth}%
 \fi\fi%
 \def\FrameCommand##1{\hskip\@totalleftmargin \hskip-\fboxsep
 \colorbox{shadecolor}{##1}\hskip-\fboxsep
     % There is no \\@totalrightmargin, so:
     \hskip-\linewidth \hskip-\@totalleftmargin \hskip\columnwidth}%
 \MakeFramed {\advance\hsize-\width
   \@totalleftmargin\z@ \linewidth\hsize
   \@setminipage}}%
 {\par\unskip\endMakeFramed%
 \at@end@of@kframe}
\makeatother

\definecolor{shadecolor}{rgb}{.97, .97, .97}
\definecolor{messagecolor}{rgb}{0, 0, 0}
\definecolor{warningcolor}{rgb}{1, 0, 1}
\definecolor{errorcolor}{rgb}{1, 0, 0}
\newenvironment{knitrout}{}{} % an empty environment to be redefined in TeX

\usepackage{alltt}   % Print solutions
%%%% comment out ONE of the above lines  
%    - the first line prints the document without the solutions, just questions
%    - the second prints the document with solutions.
\usepackage{epsfig}
\usepackage{graphicx}
\usepackage{color}
\usepackage{amsmath}
\usepackage{amssymb}
\usepackage{amsthm}
\usepackage{lscape}
\usepackage{setspace}
\usepackage{hyperref}
\usepackage{multicol}
\IfFileExists{upquote.sty}{\usepackage{upquote}}{}
\begin{document}

\singlespacing
%\onehalfspacing
%\doublespacing


\title{Assignment Covering Chapter 4 of \emph{Bayesian Computation with R}}

\author{Chris Hayduk}
\date{\today}

\maketitle



\begin{questions}
\question \textbf{Chapter 4: Multiparameter Models} Write your own code for the following:
%%%%%%%%%
\begin{parts}
\part \texttt{normchi2post()} on pg. 64
%%%%%%%%%
%\part \texttt{mycontour()} on pg. 64
%%%%%%%%%
\part \texttt{normpostsim()} on pg. 64
%%%%%%%%%
\part \texttt{rdirichlet()} on pg. 66
%%%%%%%%%
\part \texttt{beta.select()} on pg. 71
%%%%%%%%%
\part \texttt{logisticpost()} on pg. 72
%%%%%%%%%
%\part \texttt{simcontour()} on pg. 73
%%%%%%%%%
\part \texttt{howardprior()} on pg. 77
\end{parts}
%%%%%%%%%
\begin{solution}
\begin{parts}
  \part 
\begin{knitrout}\footnotesize
\definecolor{shadecolor}{rgb}{0.969, 0.969, 0.969}\color{fgcolor}\begin{kframe}
\begin{alltt}
\hlstd{normchi2post} \hlkwb{<-} \hlkwa{function}\hlstd{(}\hlkwc{theta}\hlstd{,} \hlkwc{data}\hlstd{)\{}
  \hlstd{mean} \hlkwb{<-} \hlstd{theta[}\hlnum{1}\hlstd{]}
  \hlstd{variance} \hlkwb{<-} \hlstd{theta[}\hlnum{2}\hlstd{]}

  \hlstd{y_bar} \hlkwb{<-} \hlkwd{mean}\hlstd{(data)}

  \hlstd{S} \hlkwb{<-} \hlkwd{sum}\hlstd{((data}\hlopt{-}\hlstd{y_bar)}\hlopt{^}\hlnum{2}\hlstd{)}
  \hlstd{n} \hlkwb{<-} \hlkwd{length}\hlstd{(data)}

  \hlstd{posterior_density} \hlkwb{<-} \hlnum{1}\hlopt{/}\hlstd{((variance}\hlopt{^}\hlstd{(n}\hlopt{/}\hlnum{2}\hlopt{+}\hlnum{1}\hlstd{)))} \hlopt{*}
    \hlkwd{exp}\hlstd{(}\hlopt{-}\hlnum{1}\hlopt{/}\hlstd{(}\hlnum{2}\hlopt{*}\hlstd{variance)}\hlopt{*}\hlstd{(S}\hlopt{+}\hlstd{n}\hlopt{*}\hlstd{(mean}\hlopt{-}\hlstd{y_bar)}\hlopt{^}\hlnum{2}\hlstd{))}

  \hlstd{log_density} \hlkwb{<-} \hlkwd{log}\hlstd{(posterior_density)}
  \hlcom{#print(log_density)}
  \hlkwd{return}\hlstd{(log_density)}
\hlstd{\}}

\hlcom{#Example run}

\hlkwd{data}\hlstd{(}\hlstr{"marathontimes"}\hlstd{)}
\hlkwd{attach}\hlstd{(marathontimes)}

\hlstd{d} \hlkwb{<-} \hlkwd{mycontour}\hlstd{(normchi2post,} \hlkwd{c}\hlstd{(}\hlnum{220}\hlstd{,} \hlnum{330}\hlstd{,} \hlnum{500}\hlstd{,} \hlnum{9000}\hlstd{), time,} \hlkwc{xlab}\hlstd{=}\hlstr{"mean"}\hlstd{,}
               \hlkwc{ylab}\hlstd{=}\hlstr{"variance"}\hlstd{)}
\end{alltt}
\end{kframe}
\includegraphics[width=\maxwidth]{figure/unnamed-chunk-2-1} 

\end{knitrout}
  
  \part
\begin{knitrout}\footnotesize
\definecolor{shadecolor}{rgb}{0.969, 0.969, 0.969}\color{fgcolor}\begin{kframe}
\begin{alltt}
\hlstd{normpostsim} \hlkwb{<-} \hlkwa{function}\hlstd{(}\hlkwc{data}\hlstd{,} \hlkwc{m}\hlstd{)\{}
  \hlstd{S} \hlkwb{<-} \hlkwd{sum}\hlstd{((data}\hlopt{-}\hlkwd{mean}\hlstd{(data))}\hlopt{^}\hlnum{2}\hlstd{)}

  \hlstd{n} \hlkwb{<-} \hlkwd{length}\hlstd{(data)}

  \hlstd{sigma2} \hlkwb{<-} \hlstd{S}\hlopt{/}\hlkwd{rchisq}\hlstd{(m, n}\hlopt{-}\hlnum{1}\hlstd{)}

  \hlstd{mu} \hlkwb{<-} \hlkwd{rnorm}\hlstd{(m,} \hlkwc{mean} \hlstd{=} \hlkwd{mean}\hlstd{(data),} \hlkwc{sd} \hlstd{=} \hlkwd{sqrt}\hlstd{(sigma2)}\hlopt{/}\hlkwd{sqrt}\hlstd{(n))}

  \hlstd{results} \hlkwb{<-} \hlkwd{data.frame}\hlstd{(}\hlkwc{mu} \hlstd{= mu,} \hlkwc{sigma2} \hlstd{= sigma2)}

  \hlkwd{return}\hlstd{(results)}
\hlstd{\}}

\hlcom{#Example run}
\hlstd{results} \hlkwb{<-} \hlkwd{normpostsim}\hlstd{(time,} \hlnum{1000}\hlstd{)}
\hlstd{d} \hlkwb{<-} \hlkwd{mycontour}\hlstd{(normchi2post,} \hlkwd{c}\hlstd{(}\hlnum{220}\hlstd{,} \hlnum{330}\hlstd{,} \hlnum{500}\hlstd{,} \hlnum{9000}\hlstd{), time,} \hlkwc{xlab}\hlstd{=}\hlstr{"mean"}\hlstd{,}
               \hlkwc{ylab}\hlstd{=}\hlstr{"variance"}\hlstd{)}
\hlkwd{points}\hlstd{(results}\hlopt{$}\hlstd{mu, results}\hlopt{$}\hlstd{sigma2)}
\end{alltt}
\end{kframe}
\includegraphics[width=\maxwidth]{figure/unnamed-chunk-3-1} 

\end{knitrout}
  
\end{parts}
\end{solution}
%%%%%%%%%%%%%%%%%%%%%
\question \textbf{Chapter 4: Multiparameter Models} Pg. 66 introduces the Dirichlet distribution.  Draw pdfs of this distribution for various parameter values using the functions in the R package. Find some examples on the internet of where the Dirichlet distribution is used in statistics.  
%%%%%%%%%%%%%%%%%%%%%
\question Do all CH. 4 exercises.  
%%%%%%%%%%%%%%%%%%%%%
\question \textbf{CH4, Q8 extra:} fit model using a frequentist logistic regression model; figure out how to make the equivalent confidence interval for part (d).  (Could use bootstrap, probably BC\_a and its approximation ABC, try package \texttt{\emph{bootBCa}}; include a histogram of your BC\_a bootstrap resampled probability values and normal quantile plot of those values.  Let number of bootstrap samples be $B=9,999$.)
\end{questions}


\end{document}
