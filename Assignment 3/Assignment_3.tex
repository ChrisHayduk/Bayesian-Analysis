%\documentclass[12pt,addpoints]{exam}   % Print w/o solutions
\documentclass[12pt,addpoints,answers]{exam}\usepackage[]{graphicx}\usepackage[]{color}
%% maxwidth is the original width if it is less than linewidth
%% otherwise use linewidth (to make sure the graphics do not exceed the margin)
\makeatletter
\def\maxwidth{ %
  \ifdim\Gin@nat@width>\linewidth
    \linewidth
  \else
    \Gin@nat@width
  \fi
}
\makeatother

\definecolor{fgcolor}{rgb}{0.345, 0.345, 0.345}
\newcommand{\hlnum}[1]{\textcolor[rgb]{0.686,0.059,0.569}{#1}}%
\newcommand{\hlstr}[1]{\textcolor[rgb]{0.192,0.494,0.8}{#1}}%
\newcommand{\hlcom}[1]{\textcolor[rgb]{0.678,0.584,0.686}{\textit{#1}}}%
\newcommand{\hlopt}[1]{\textcolor[rgb]{0,0,0}{#1}}%
\newcommand{\hlstd}[1]{\textcolor[rgb]{0.345,0.345,0.345}{#1}}%
\newcommand{\hlkwa}[1]{\textcolor[rgb]{0.161,0.373,0.58}{\textbf{#1}}}%
\newcommand{\hlkwb}[1]{\textcolor[rgb]{0.69,0.353,0.396}{#1}}%
\newcommand{\hlkwc}[1]{\textcolor[rgb]{0.333,0.667,0.333}{#1}}%
\newcommand{\hlkwd}[1]{\textcolor[rgb]{0.737,0.353,0.396}{\textbf{#1}}}%
\let\hlipl\hlkwb

\usepackage{framed}
\makeatletter
\newenvironment{kframe}{%
 \def\at@end@of@kframe{}%
 \ifinner\ifhmode%
  \def\at@end@of@kframe{\end{minipage}}%
  \begin{minipage}{\columnwidth}%
 \fi\fi%
 \def\FrameCommand##1{\hskip\@totalleftmargin \hskip-\fboxsep
 \colorbox{shadecolor}{##1}\hskip-\fboxsep
     % There is no \\@totalrightmargin, so:
     \hskip-\linewidth \hskip-\@totalleftmargin \hskip\columnwidth}%
 \MakeFramed {\advance\hsize-\width
   \@totalleftmargin\z@ \linewidth\hsize
   \@setminipage}}%
 {\par\unskip\endMakeFramed%
 \at@end@of@kframe}
\makeatother

\definecolor{shadecolor}{rgb}{.97, .97, .97}
\definecolor{messagecolor}{rgb}{0, 0, 0}
\definecolor{warningcolor}{rgb}{1, 0, 1}
\definecolor{errorcolor}{rgb}{1, 0, 0}
\newenvironment{knitrout}{}{} % an empty environment to be redefined in TeX

\usepackage{alltt}   % Print solutions
\usepackage{epsfig}
\usepackage{graphicx}
\usepackage{color}
\usepackage{amsmath}
\usepackage{amssymb}
\usepackage{amsthm}
\usepackage{lscape}
\usepackage{setspace}
\usepackage{hyperref}
\usepackage{multicol}
\IfFileExists{upquote.sty}{\usepackage{upquote}}{}
\begin{document}

\singlespacing
%\onehalfspacing
%\doublespacing


\title{Ch. 4 Extra Exercises: Part 2}

\author{}
\date{\today}

\maketitle





% \noindent 

\begin{questions}
\question On pg. 90 of CH 4 in  \emph{Statistical Rethinking}, the book uses a function \texttt{cov2cor()} to convert a covariance matrix into a correlation matrix.    Write your own function which does the same and write out the computations using matrix algebra notation.
\begin{solution}
\begin{knitrout}\footnotesize
\definecolor{shadecolor}{rgb}{0.969, 0.969, 0.969}\color{fgcolor}\begin{kframe}
\begin{alltt}
\hlstd{my_cov2cor} \hlkwb{<-} \hlkwa{function}\hlstd{(}\hlkwc{cov_mat}\hlstd{)\{}
  \hlstd{diag} \hlkwb{<-} \hlkwd{diag}\hlstd{(cov_mat)}
  \hlstd{diag_inv_sqrt} \hlkwb{<-} \hlstd{(diag)}\hlopt{^}\hlstd{(}\hlopt{-}\hlnum{1}\hlopt{/}\hlnum{2}\hlstd{)}
  \hlstd{corr_mat} \hlkwb{<-} \hlstd{diag_inv_sqrt} \hlopt{*} \hlstd{cov_mat} \hlopt{*} \hlkwd{rep}\hlstd{(diag_inv_sqrt,} \hlkwc{each} \hlstd{=} \hlkwd{dim}\hlstd{(cov_mat)[}\hlnum{1L}\hlstd{])}

  \hlkwd{return}\hlstd{(corr_mat)}
\hlstd{\}}
\end{alltt}
\end{kframe}
\end{knitrout}
\newpage
Now let's compare the output of this function to the built-in \texttt{cov2cor()} function:
\begin{knitrout}\footnotesize
\definecolor{shadecolor}{rgb}{0.969, 0.969, 0.969}\color{fgcolor}\begin{kframe}
\begin{alltt}
\hlcom{#Test run using data from Ch. 4 pg. 87-90}
\hlkwd{data}\hlstd{(Howell1)}
\hlstd{d} \hlkwb{<-} \hlstd{Howell1}
\hlstd{d2} \hlkwb{<-} \hlstd{d[d}\hlopt{$}\hlstd{age} \hlopt{>=} \hlnum{18}\hlstd{,]}

\hlstd{flist} \hlkwb{<-} \hlkwd{alist}\hlstd{(}
  \hlstd{height} \hlopt{~} \hlkwd{dnorm}\hlstd{(mu, sigma),}
  \hlstd{mu} \hlopt{~} \hlkwd{dnorm}\hlstd{(}\hlnum{178}\hlstd{,} \hlnum{20}\hlstd{),}
  \hlstd{sigma} \hlopt{~} \hlkwd{dunif}\hlstd{(}\hlnum{0}\hlstd{,} \hlnum{50}\hlstd{)}
\hlstd{)}

\hlstd{m4.1} \hlkwb{<-} \hlkwd{map}\hlstd{(flist,} \hlkwc{data}\hlstd{=d2)}

\hlkwd{cov2cor}\hlstd{(}\hlkwd{vcov}\hlstd{(m4.1))}
\end{alltt}
\begin{verbatim}
##                mu       sigma
## mu    1.000000000 0.001816912
## sigma 0.001816912 1.000000000
\end{verbatim}
\begin{alltt}
\hlkwd{my_cov2cor}\hlstd{(}\hlkwd{vcov}\hlstd{(m4.1))}
\end{alltt}
\begin{verbatim}
##                mu       sigma
## mu    1.000000000 0.001816912
## sigma 0.001816912 1.000000000
\end{verbatim}
\end{kframe}
\end{knitrout}
\end{solution}
%%%%%%%%%%%%%%%%%%%%%%%%
\question On pg. 92 of CH 4 in  \emph{Statistical Rethinking}, the author talks about regression to the mean and shrinkage.  Read more about shrinkage estimation (including those articles I gave you last week), to explain more about the purpose and benefit of using shrinkage estimation.  Include citations for any references you use. 
\begin{solution}

Shrinkage estimation generallly reduces the variance in parameter estimation. This may serve to improve out-of-sample prediction by reducing the variance of the model while increasing the bias. If the "correct" parameters for a particular shrinkage method are chosen, then the trade-off between variance and bias should result in a model with better predictive capability when compared with an OLS model.

Ridge regression, one such shrinkage method, is very similar to OLS regression. It includes a penalty term with the RSS calculation which penalizes the model for having large $\beta$ values. The parameter $\lambda \geq 0$ controls the size of the penalty, with $\lambda = 0$ simplifying the model to OLS regression. The penalty term shrinks the $\beta$ terms towards 0 (but never reaching 0).

Lasso regression is another shrinkage method which decreases the variance of the $\beta$ values. Lasso minimizes the RSS, but includes the constraint that $\sum\limits_{j=1}^{p} |\beta_j| \leq t$. If $t$ is chosen sufficiently small, then some of the $\beta$ values will be reduced all the way to 0. In this manner, lasso regression is able to perform a sort of continuous subset selection.

\begin{thebibliography}{9}
\bibitem{islr}
James, Gareth, et al. \textit{An Introduction to Statistical Learning: with Applications in R}. Springer, 2017.

\bibitem{eslr}
Hastie, Trevor, et al. \textit{The Elements of Statistical Learning}. Springer, 2004.

\bibitem{regmodeling}
Harrell, Frank. \textit{Regression Modeling Strategies}. Springer, 2016.

\bibitem{brown}
Brown, Lawrence. \textit{A Unified View of Regression, Shrinkage, Empirical Bayes, Hierarchical Bayes, and Random Effects}. Oberwolfach Reports, 2007.
\end{thebibliography}
\end{solution}
%%%%%%%%%%%%%%%%%%%%%%%%
\question On pg. 99 of CH 4 in \emph{Statistical Rethinking}, the author writes, ``But in more complex models, strong [parameter] correlations like this can make it difficult to fit the model to the data.''
  \begin{parts}
  \part Explain why correlations between pairs of parameters is a problem.  
  %%%%%%
  \part Why does the author only center the $x$-variable, \texttt{weight}, and not the $y$-variable \texttt{height}?
  %%%%%%
  \end{parts}
\begin{solution}
  \begin{parts}
  \part Let's try calculating the variance-covariance matrix to see why correlation between parameter estimates can be problematic:
\begin{knitrout}\footnotesize
\definecolor{shadecolor}{rgb}{0.969, 0.969, 0.969}\color{fgcolor}\begin{kframe}
\begin{alltt}
\hlcom{#Create model}
\hlstd{model} \hlkwb{<-} \hlkwd{lm}\hlstd{(height} \hlopt{~} \hlstd{age} \hlopt{+} \hlstd{weight,} \hlkwc{data} \hlstd{= d2)}
\hlkwd{cov2cor}\hlstd{(}\hlkwd{vcov}\hlstd{(model))}
\end{alltt}
\begin{verbatim}
##             (Intercept)        age     weight
## (Intercept)   1.0000000 -0.4787928 -0.9387455
## age          -0.4787928  1.0000000  0.1729043
## weight       -0.9387455  0.1729043  1.0000000
\end{verbatim}
\begin{alltt}
\hlcom{#Calculate variance-covariance matrix by marix algebra}
\hlstd{X} \hlkwb{<-} \hlkwd{data.frame}\hlstd{(}\hlkwc{age} \hlstd{= d2}\hlopt{$}\hlstd{age,} \hlkwc{weight} \hlstd{= d2}\hlopt{$}\hlstd{weight)}
\hlstd{X} \hlkwb{<-} \hlkwd{data.matrix}\hlstd{(X)}
\hlstd{Y} \hlkwb{<-} \hlkwd{data.matrix}\hlstd{(d2}\hlopt{$}\hlstd{height)}

\hlstd{n} \hlkwb{<-} \hlkwd{nrow}\hlstd{(X)}
\hlstd{p} \hlkwb{<-} \hlkwd{length}\hlstd{(model}\hlopt{$}\hlstd{coefficients)}

\hlstd{I} \hlkwb{<-} \hlkwd{diag}\hlstd{(n)}

\hlstd{C} \hlkwb{<-} \hlkwd{solve}\hlstd{((}\hlkwd{t}\hlstd{(X)} \hlopt \hlstd{X))}
\hlstd{P} \hlkwb{<-} \hlstd{X} \hlopt \hlstd{C} \hlopt \hlkwd{t}\hlstd{(X)}

\hlstd{var} \hlkwb{<-} \hlstd{(}\hlkwd{t}\hlstd{(Y)} \hlopt \hlstd{(I}\hlopt{-}\hlstd{P)} \hlopt \hlstd{Y)}\hlopt{/}\hlstd{(n} \hlopt{-} \hlstd{p} \hlopt{-} \hlnum{1}\hlstd{)}

\hlcom{#Variance-covariance matrix for model}
\hlkwd{vcov}\hlstd{(model)}
\end{alltt}
\begin{verbatim}
##             (Intercept)           age        weight
## (Intercept)  4.74264415 -0.0180072460 -0.0873138157
## age         -0.01800725  0.0002982488  0.0001275322
## weight      -0.08731382  0.0001275322  0.0018241041
\end{verbatim}
\begin{alltt}
\hlcom{#Unbiased estimator of variance-covariance matrix}
\hlstd{var[}\hlnum{1}\hlstd{]}\hlopt{*}\hlstd{C}
\end{alltt}
\begin{verbatim}
##                 age       weight
## age     0.002009493 -0.001783175
## weight -0.001783175  0.001893643
\end{verbatim}
\end{kframe}
\end{knitrout}
  
  \part Centering the $x$-variable but not the $y$-variable retains the estimate and interpretability of the $\beta$ coefficient and improves the interpretability of the intercept $\alpha$. The intercept is now interpreted as the value of $\hat{y}$ when $x$ is equal to the mean of the data. This ensures that the intercept is of practical importance, as it does not make sense to consider the case where $x = 0$ in many regression equations.
  \end{parts}
\end{solution}
%%%%%%%%%%%%%%%%%%%%%%%%
\question Use the \texttt{d2} data from the chapter to answer the following questions, which are extensions of \textbf{4H2}:
\begin{parts}
\part Fit a frequentist simple linear regression with \texttt{weight} as the explanatory variable and \texttt{height} as the response variable.  Make a scatterplot of the data and plot both the frequentist and Bayesian lines.  Is there much of a difference?
  %%%%%%
\part Check the regression assumptions on your frequentist model.  Include any relevant tables/graphs with your assumption checks.
  %%%%%%
  \part Are correlations between pairs of parameters a problem in frequentist regression?  Calculate the variance-covariance matrix for $\boldsymbol{\hat{\beta}}=(\hat{\beta}_0,\hat{\beta}_1)$ using the formulas in the appendix to CH 3 of \emph{Regression Analysis by Example} (your regression textbook).
  %%%%%%
  \part Does centering \texttt{weight} make a difference in the correlations between parameters in frequentist regression?  Does it make a difference in the parameter estimates themselves?   Try using the function \texttt{scale()} to center \texttt{weight} in your code.  
  %%%%%%
  \part Use the function \texttt{confint()} to create a 95\% confidence interval of the slope for your frequentist regression.  Calculate a 95\% posterior probability interval for the slope for your Bayesian regression.  Interpret both and explain the difference between the two.
  %%%%%%
  \part Create a figure with 4 scatterplots, each with \texttt{weight} on the $x$-axis and \texttt{height} on the $y$-axis.  On the first plot, add the Bayesian fitted model and add the 95\% HPDI intervals (like Fig. 4.8).  On the second plot, add the Bayesian fitted model and add the 95\% PI intervals (also like Fig. 4.8).  On the third plot, add the frequentist fitted model and the 95\% confidence intervals for prediction (like slide 35 in Lecture 3 of the regression class).  Finally, on the fourth plot, add the frequentist fitted model and the 95\% prediction intervals for prediction (also like slide 35 in Lecture 3 of the regression class).  Interpret all four plots and explain the connections between them.  
\end{parts}
%%%%%%%%%%%%%%%%%%%%%%%%
\end{questions}
\begin{solution}
\begin{parts}
  \part 
\begin{knitrout}\footnotesize
\definecolor{shadecolor}{rgb}{0.969, 0.969, 0.969}\color{fgcolor}\begin{kframe}
\begin{alltt}
\hlstd{lm.1} \hlkwb{<-} \hlkwd{lm}\hlstd{(height} \hlopt{~} \hlstd{weight,} \hlkwc{data} \hlstd{= d2)}

\hlkwd{summary}\hlstd{(lm.1)}
\end{alltt}
\begin{verbatim}
## 
## Call:
## lm(formula = height ~ weight, data = d2)
## 
## Residuals:
##      Min       1Q   Median       3Q      Max 
## -19.7464  -2.8835   0.0222   3.1424  14.7744 
## 
## Coefficients:
##              Estimate Std. Error t value Pr(>|t|)    
## (Intercept) 113.87939    1.91107   59.59   <2e-16 ***
## weight        0.90503    0.04205   21.52   <2e-16 ***
## ---
## Signif. codes:  0 '***' 0.001 '**' 0.01 '*' 0.05 '.' 0.1 ' ' 1
## 
## Residual standard error: 5.086 on 350 degrees of freedom
## Multiple R-squared:  0.5696,	Adjusted R-squared:  0.5684 
## F-statistic: 463.3 on 1 and 350 DF,  p-value: < 2.2e-16
\end{verbatim}
\end{kframe}
\end{knitrout}

\begin{knitrout}\footnotesize
\definecolor{shadecolor}{rgb}{0.969, 0.969, 0.969}\color{fgcolor}\begin{kframe}
\begin{alltt}
\hlstd{m4.3} \hlkwb{<-} \hlkwd{map}\hlstd{(}
  \hlkwd{alist}\hlstd{(}
    \hlstd{height} \hlopt{~} \hlkwd{dnorm}\hlstd{(mu, sigma),}
    \hlstd{mu} \hlkwb{<-} \hlstd{a} \hlopt{+} \hlstd{b}\hlopt{*}\hlstd{weight,}
    \hlstd{a} \hlopt{~} \hlkwd{dnorm}\hlstd{(}\hlnum{156}\hlstd{,} \hlnum{100}\hlstd{),}
    \hlstd{b} \hlopt{~} \hlkwd{dnorm}\hlstd{(}\hlnum{0}\hlstd{,} \hlnum{10}\hlstd{),}
    \hlstd{sigma} \hlopt{~} \hlkwd{dunif}\hlstd{(}\hlnum{0}\hlstd{,} \hlnum{50}\hlstd{)}
  \hlstd{),}
\hlkwc{data} \hlstd{= d2)}

\hlkwd{precis}\hlstd{(m4.3)}
\end{alltt}
\begin{verbatim}
##         Mean StdDev   5.5%  94.5%
## a     113.90   1.91 110.85 116.94
## b       0.90   0.04   0.84   0.97
## sigma   5.07   0.19   4.77   5.38
\end{verbatim}
\end{kframe}
\end{knitrout}
  
\begin{knitrout}\footnotesize
\definecolor{shadecolor}{rgb}{0.969, 0.969, 0.969}\color{fgcolor}\begin{kframe}
\begin{alltt}
\hlkwd{ggplot}\hlstd{(}\hlkwc{data} \hlstd{= d2,}\hlkwc{mapping} \hlstd{=} \hlkwd{aes}\hlstd{(}\hlkwc{x} \hlstd{= weight,}\hlkwc{y} \hlstd{= height))} \hlopt{+}
  \hlkwd{geom_point}\hlstd{()} \hlopt{+}
  \hlkwd{geom_abline}\hlstd{(}\hlkwd{aes}\hlstd{(}\hlkwc{color} \hlstd{=} \hlstr{"Frequentist Model"}\hlstd{,} \hlkwc{slope} \hlstd{= lm.1}\hlopt{$}\hlstd{coefficients[}\hlnum{2}\hlstd{],}
              \hlkwc{intercept} \hlstd{= lm.1}\hlopt{$}\hlstd{coefficients[}\hlnum{1}\hlstd{]),} \hlkwc{size} \hlstd{=} \hlnum{1.5}\hlstd{)} \hlopt{+}
  \hlkwd{geom_abline}\hlstd{(}\hlkwd{aes}\hlstd{(}\hlkwc{color} \hlstd{=} \hlstr{"Bayesian Model"}\hlstd{,} \hlkwc{slope} \hlstd{=} \hlkwd{coef}\hlstd{(m4.3)[}\hlstr{"b"}\hlstd{],}
              \hlkwc{intercept} \hlstd{=} \hlkwd{coef}\hlstd{(m4.3)[}\hlstr{"a"}\hlstd{]),} \hlkwc{size} \hlstd{=} \hlnum{1.5}\hlstd{)} \hlopt{+}
  \hlkwd{ggtitle}\hlstd{(}\hlstr{"Frquentist Model vs. Bayesian Model"}\hlstd{)} \hlopt{+}
  \hlkwd{labs}\hlstd{(}\hlkwc{color} \hlstd{=} \hlstr{"type of model"}\hlstd{)} \hlopt{+}
  \hlkwd{scale_color_manual}\hlstd{(}\hlkwc{values} \hlstd{=} \hlkwd{c}\hlstd{(}\hlstr{"firebrick"}\hlstd{,} \hlstr{"navyblue"}\hlstd{))} \hlopt{+}
  \hlstd{theme.info}
\end{alltt}
\end{kframe}
\includegraphics[width=\maxwidth]{figure/unnamed-chunk-7-1} 

\end{knitrout}
The two regression lines are essentially identical.
\newpage
  \part Regression assumptions:
  \begin{enumerate}
    \item Fixed $x$ with no measurement error:\\
    The $x$ values are not fixed because we observed each weight value; they were not assigned before the regression. They are also not measured without error as any scale will likely introduce a small amount of measurement noise. However, on a well-calibrated scale, the amount of measurement noice introduced is likely ignorable.
    \item $x$ and $y$ linearly related:\\
    We can see from the plot above that $x$ and $y$ are clearly linearly related.
    \item Constant variance:\\
\begin{knitrout}\footnotesize
\definecolor{shadecolor}{rgb}{0.969, 0.969, 0.969}\color{fgcolor}\begin{kframe}
\begin{alltt}
\hlstd{y.augment} \hlkwb{<-} \hlkwd{augment}\hlstd{(lm.1)}
\hlstd{y.augment}
\end{alltt}
\begin{verbatim}
## # A tibble: 352 x 10
##    .rownames height weight .fitted .se.fit .resid    .hat .sigma .cooksd
##    <chr>      <dbl>  <dbl>   <dbl>   <dbl>  <dbl>   <dbl>  <dbl>   <dbl>
##  1 1           152.   47.8    157.   0.296 -5.40  0.00339   5.09 1.92e-3
##  2 2           140.   36.5    147.   0.449 -7.20  0.00778   5.08 7.92e-3
##  3 3           137.   31.9    143.   0.615 -6.19  0.0146    5.08 1.12e-2
##  4 4           157.   53.0    162.   0.434 -5.04  0.00727   5.09 3.62e-3
##  5 5           145.   41.3    151.   0.313 -5.82  0.00378   5.08 2.50e-3
##  6 6           164.   63.0    171.   0.804 -7.06  0.0250    5.08 2.53e-2
##  7 7           149.   38.2    148.   0.392  0.734 0.00595   5.09 6.27e-5
##  8 8           169.   55.5    164.   0.518  4.82  0.0104    5.09 4.75e-3
##  9 9           148.   34.9    145.   0.505  2.52  0.00984   5.09 1.23e-3
## 10 10          165.   54.5    163.   0.483  1.91  0.00900   5.09 6.45e-4
## # ... with 342 more rows, and 1 more variable: .std.resid <dbl>
\end{verbatim}
\begin{alltt}
\hlstd{y.augment} \hlopt \hlkwd{ggplot}\hlstd{(}\hlkwd{aes}\hlstd{(}\hlkwc{x}\hlstd{=weight,} \hlkwc{y}\hlstd{=.resid))} \hlopt{+}
  \hlkwd{geom_point}\hlstd{(}\hlkwc{color}\hlstd{=}\hlstr{"#99000070"}\hlstd{,} \hlkwc{size}\hlstd{=}\hlnum{2}\hlstd{)} \hlopt{+}
  \hlkwd{ggtitle}\hlstd{(}\hlstr{"Residual Plot"}\hlstd{)} \hlopt{+}
  \hlkwd{labs}\hlstd{(}\hlkwc{x}\hlstd{=}\hlstr{"Weight (kg)"}\hlstd{,} \hlkwc{y}\hlstd{=}\hlstr{"Residuals"}\hlstd{)} \hlopt{+}
  \hlkwd{geom_hline}\hlstd{(}\hlkwc{yintercept}\hlstd{=}\hlnum{0}\hlstd{,} \hlkwc{color}\hlstd{=}\hlstr{"gray50"}\hlstd{,}  \hlkwc{lty}\hlstd{=}\hlnum{2}\hlstd{)} \hlopt{+}
  \hlstd{theme.info}
\end{alltt}
\end{kframe}
\includegraphics[width=\maxwidth]{figure/unnamed-chunk-8-1} 

\end{knitrout}
  We can see from the plot above that the variance in the residuals is constant regardless of the individual's weight.
  \item Mean of the residuals = 0:\\
\begin{knitrout}\footnotesize
\definecolor{shadecolor}{rgb}{0.969, 0.969, 0.969}\color{fgcolor}\begin{kframe}
\begin{alltt}
\hlkwd{mean}\hlstd{(y.augment}\hlopt{$}\hlstd{.resid)}
\end{alltt}
\begin{verbatim}
## [1] 5.337416e-17
\end{verbatim}
\end{kframe}
\end{knitrout}
  As you can see, the mean of the residuals is very close to 0.
  \item Normally distributed errors:\\
\begin{knitrout}\footnotesize
\definecolor{shadecolor}{rgb}{0.969, 0.969, 0.969}\color{fgcolor}\begin{kframe}
\begin{alltt}
\hlstd{y.augment} \hlopt  \hlkwd{ggplot}\hlstd{(}\hlkwd{aes}\hlstd{(}\hlkwc{sample}\hlstd{=.resid))} \hlopt{+}
  \hlkwd{stat_qq}\hlstd{(}\hlkwc{col} \hlstd{=} \hlstr{"firebrick"}\hlstd{)} \hlopt{+}
  \hlkwd{stat_qq_line}\hlstd{()} \hlopt{+}
  \hlkwd{ggtitle}\hlstd{(}\hlstr{"Normal Quantile Plot"}\hlstd{)} \hlopt{+}
  \hlstd{theme.info}
\end{alltt}
\end{kframe}
\includegraphics[width=\maxwidth]{figure/unnamed-chunk-10-1} 

\end{knitrout}
  The Normal Quantile Plot shows a rougly straight line. There is some deviation towards each end of the line, indicating that the data comes from a heavy-tailed distribution. However, the deviation is small enough that we may still be able to run our regression with minimal error.
  \end{enumerate}
  \part 
\begin{knitrout}\footnotesize
\definecolor{shadecolor}{rgb}{0.969, 0.969, 0.969}\color{fgcolor}\begin{kframe}
\begin{alltt}
\hlstd{calculate_variance_covariance_matrix} \hlkwb{<-} \hlkwa{function}\hlstd{(}\hlkwc{x_mat}\hlstd{,} \hlkwc{y_mat}\hlstd{)\{}
  \hlstd{x_var} \hlkwb{<-} \hlkwd{as.matrix}\hlstd{(x_mat)}
  \hlstd{intercept} \hlkwb{<-} \hlkwd{matrix}\hlstd{(}\hlkwd{rep}\hlstd{(}\hlnum{1}\hlstd{,} \hlkwd{nrow}\hlstd{(x_var)),} \hlkwc{ncol} \hlstd{=} \hlnum{1}\hlstd{)}

  \hlstd{X} \hlkwb{<-} \hlkwd{cbind}\hlstd{(intercept, x_var)}

  \hlstd{Y} \hlkwb{<-} \hlkwd{as.matrix}\hlstd{(y_mat)}

  \hlstd{C} \hlkwb{<-} \hlkwd{solve}\hlstd{(}\hlkwd{t}\hlstd{(X)} \hlopt \hlstd{X)}

  \hlstd{beta_hat} \hlkwb{<-} \hlstd{C} \hlopt \hlkwd{t}\hlstd{(X)} \hlopt \hlstd{Y}

  \hlstd{var_covar} \hlkwb{<-} \hlstd{C} \hlopt{*} \hlkwd{var}\hlstd{(Y)[}\hlnum{1}\hlstd{]}

  \hlkwd{return}\hlstd{(var_covar)}
\hlstd{\}}

\hlkwd{print}\hlstd{(}\hlkwd{calculate_variance_covariance_matrix}\hlstd{(d2}\hlopt{$}\hlstd{weight, d2}\hlopt{$}\hlstd{height))}
\end{alltt}
\begin{verbatim}
##           [,1]         [,2]
## [1,]  8.462223 -0.184304039
## [2,] -0.184304  0.004096511
\end{verbatim}
\end{kframe}
\end{knitrout}
  \part
\begin{knitrout}\footnotesize
\definecolor{shadecolor}{rgb}{0.969, 0.969, 0.969}\color{fgcolor}\begin{kframe}
\begin{alltt}
\hlstd{d2}\hlopt{$}\hlstd{centered_weight} \hlkwb{<-} \hlkwd{scale}\hlstd{(d2}\hlopt{$}\hlstd{weight,} \hlkwc{scale} \hlstd{=} \hlnum{FALSE}\hlstd{)}

\hlstd{lm.2} \hlkwb{<-} \hlkwd{lm}\hlstd{(height} \hlopt{~} \hlstd{centered_weight,} \hlkwc{data} \hlstd{= d2)}

\hlkwd{summary}\hlstd{(lm.2)}
\end{alltt}
\begin{verbatim}
## 
## Call:
## lm(formula = height ~ centered_weight, data = d2)
## 
## Residuals:
##      Min       1Q   Median       3Q      Max 
## -19.7464  -2.8835   0.0222   3.1424  14.7744 
## 
## Coefficients:
##                  Estimate Std. Error t value Pr(>|t|)    
## (Intercept)     154.59709    0.27110  570.25   <2e-16 ***
## centered_weight   0.90503    0.04205   21.52   <2e-16 ***
## ---
## Signif. codes:  0 '***' 0.001 '**' 0.01 '*' 0.05 '.' 0.1 ' ' 1
## 
## Residual standard error: 5.086 on 350 degrees of freedom
## Multiple R-squared:  0.5696,	Adjusted R-squared:  0.5684 
## F-statistic: 463.3 on 1 and 350 DF,  p-value: < 2.2e-16
\end{verbatim}
\begin{alltt}
\hlkwd{print}\hlstd{(}\hlkwd{calculate_variance_covariance_matrix}\hlstd{(d2}\hlopt{$}\hlstd{centered_weight, d2}\hlopt{$}\hlstd{height))}
\end{alltt}
\begin{verbatim}
##               [,1]          [,2]
## [1,]  1.702946e-01 -3.638433e-18
## [2,] -3.638433e-18  4.096511e-03
\end{verbatim}
\end{kframe}
\end{knitrout}
  Each entry in the variance-covariance matrix has been altered. In the regression output, we can see that the intercept has changed while the slope has remained the same.
  \part
\begin{knitrout}\footnotesize
\definecolor{shadecolor}{rgb}{0.969, 0.969, 0.969}\color{fgcolor}\begin{kframe}
\begin{alltt}
\hlcom{#Frequentist regression confidence interval for the slope}
\hlkwd{confint}\hlstd{(lm.1)}
\end{alltt}
\begin{verbatim}
##                   2.5 %      97.5 %
## (Intercept) 110.1207774 117.6380098
## weight        0.8223315   0.9877267
\end{verbatim}
\begin{alltt}
\hlkwd{confint}\hlstd{(lm.2)}
\end{alltt}
\begin{verbatim}
##                       2.5 %      97.5 %
## (Intercept)     154.0638974 155.1302878
## centered_weight   0.8223315   0.9877267
\end{verbatim}
\end{kframe}
\end{knitrout}

\begin{knitrout}\footnotesize
\definecolor{shadecolor}{rgb}{0.969, 0.969, 0.969}\color{fgcolor}\begin{kframe}
\begin{alltt}
\hlcom{#Bayesian regression posterior probability interval for the slope}
\hlstd{post} \hlkwb{<-} \hlkwd{extract.samples}\hlstd{(m4.3,} \hlkwc{n}\hlstd{=}\hlnum{1e4}\hlstd{)}

\hlkwd{precis}\hlstd{(post,} \hlkwc{prob} \hlstd{=} \hlnum{0.95}\hlstd{,} \hlkwc{digits} \hlstd{=} \hlnum{5}\hlstd{)}
\end{alltt}
\begin{verbatim}
##            Mean  StdDev     |0.95     0.95|
## a     113.87267 1.88844 110.27727 117.64641
## b       0.90505 0.04152   0.82278   0.98419
## sigma   5.07377 0.19095   4.68890   5.44710
\end{verbatim}
\end{kframe}
\end{knitrout}

The 95\% confidence interval for the slope can be thought of in this manner: if we were to take 100 different samples from the population, we would expect 95 of the regressions to yield an estimate for $\hat{\beta}_1$ that falls within the 95\% confidence interval.

The 95\% posterior probability interval for the slope represents the range within which we expect 95 out of 100 slopes sampled from the posterior distribution to fall.

The difference between the two lies within the interpretation of the slope parameter. Frequentist regression views $\hat{\beta}_1$ as a fixed parameter estimated from random data. However, Bayesian regression views $\hat{\beta}_1$ as a random variable with a sampling distribution (the posterior probability distribution).

\begin{knitrout}\footnotesize
\definecolor{shadecolor}{rgb}{0.969, 0.969, 0.969}\color{fgcolor}\begin{kframe}
\begin{alltt}
\hlcom{#Bootstrap interval}
\hlstd{beta_hat} \hlkwb{<-} \hlkwd{rep}\hlstd{(}\hlnum{0}\hlstd{,} \hlkwc{times} \hlstd{=} \hlnum{10000}\hlstd{)}
\hlkwa{for}\hlstd{(i} \hlkwa{in} \hlnum{1}\hlopt{:}\hlnum{10000}\hlstd{)\{}
  \hlstd{data} \hlkwb{<-} \hlkwd{sample_n}\hlstd{(d2,} \hlkwc{size} \hlstd{=} \hlkwd{length}\hlstd{(d2}\hlopt{$}\hlstd{height),} \hlkwc{replace} \hlstd{=} \hlnum{TRUE}\hlstd{)}
  \hlstd{lm} \hlkwb{<-} \hlkwd{lm}\hlstd{(height} \hlopt{~} \hlstd{weight,} \hlkwc{data} \hlstd{= data)}
  \hlstd{beta_hat[i]} \hlkwb{<-} \hlstd{lm}\hlopt{$}\hlstd{coefficients[}\hlnum{2}\hlstd{]}
\hlstd{\}}

\hlstd{data} \hlkwb{<-} \hlkwd{data.frame}\hlstd{(}\hlstr{"beta_hat"} \hlstd{= beta_hat)}

\hlkwd{ggplot}\hlstd{(data,} \hlkwd{aes}\hlstd{(}\hlkwc{x}\hlstd{=beta_hat))} \hlopt{+}
  \hlkwd{geom_histogram}\hlstd{(}\hlkwd{aes}\hlstd{(}\hlkwc{y} \hlstd{= ..density..),} \hlkwc{bins} \hlstd{=} \hlnum{50}\hlstd{,}
                 \hlkwc{colour} \hlstd{=} \hlstr{"black"}\hlstd{,}
                   \hlkwc{fill} \hlstd{=} \hlstr{"white"}\hlstd{)} \hlopt{+}
  \hlkwd{ggtitle}\hlstd{(}\hlstr{"Histogram of Bootstrap Estimates for the Slope Parameter"}\hlstd{)} \hlopt{+}
  \hlkwd{labs}\hlstd{(}\hlkwc{x} \hlstd{=} \hlstr{"Slope Estimate"}\hlstd{,} \hlkwc{y} \hlstd{=} \hlstr{"Density"}\hlstd{)} \hlopt{+}
  \hlkwd{stat_function}\hlstd{(}\hlkwc{fun} \hlstd{= dnorm,} \hlkwc{n} \hlstd{=} \hlnum{101}\hlstd{,} \hlkwc{args} \hlstd{=} \hlkwd{list}\hlstd{(}\hlkwc{mean} \hlstd{=} \hlkwd{mean}\hlstd{(beta_hat),}
                                                  \hlkwc{sd} \hlstd{=} \hlkwd{sd}\hlstd{(beta_hat)))} \hlopt{+}
  \hlkwd{geom_vline}\hlstd{(}\hlkwc{xintercept} \hlstd{=} \hlkwd{mean}\hlstd{(beta_hat),} \hlkwc{linetype} \hlstd{=} \hlstr{"dashed"}\hlstd{,}
             \hlkwc{color} \hlstd{=} \hlstr{"red"}\hlstd{,} \hlkwc{size} \hlstd{=} \hlnum{1.5}\hlstd{)} \hlopt{+}
  \hlstd{theme.info}
\end{alltt}
\end{kframe}
\includegraphics[width=\maxwidth]{figure/unnamed-chunk-15-1} 

\end{knitrout}
\newpage
\part
Let's plot the confidence \& prediction intervals for both the frequentist and Bayesian models:\\
\begin{knitrout}\footnotesize
\definecolor{shadecolor}{rgb}{0.969, 0.969, 0.969}\color{fgcolor}
\includegraphics[width=\maxwidth]{figure/unnamed-chunk-16-1} 

\end{knitrout}

The plot in the top left demonstrates the densest 95\% interval for the regression line.

The plot in the top right demonstrates the 95\% confidence interval for a new $y$ given each $x$ value.

The bottom left plot demonstrates the 95\% confidence interval for the average $y$ value as $x$ varies.

The bottom right plot demonstrates the 95\% prediction interval for a new $y$ given each $x$ value.

The HPDI and Confidence Interval plots are relaying roughly the same information. They each specify that the average $y$ value for a given $x$ should fall within that region with 95\% probability.

The PI and Prediction Interval plots both show the region where the model expects to find 95\% of heights in the population at each weight.
\end{parts}
\end{solution}


\end{document}
