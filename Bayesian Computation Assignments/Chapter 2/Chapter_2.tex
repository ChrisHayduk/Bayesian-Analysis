\documentclass[12pt,addpoints,answers]{exam}\usepackage[]{graphicx}\usepackage[]{color}
%% maxwidth is the original width if it is less than linewidth
%% otherwise use linewidth (to make sure the graphics do not exceed the margin)
\makeatletter
\def\maxwidth{ %
  \ifdim\Gin@nat@width>\linewidth
    \linewidth
  \else
    \Gin@nat@width
  \fi
}
\makeatother

\definecolor{fgcolor}{rgb}{0.345, 0.345, 0.345}
\newcommand{\hlnum}[1]{\textcolor[rgb]{0.686,0.059,0.569}{#1}}%
\newcommand{\hlstr}[1]{\textcolor[rgb]{0.192,0.494,0.8}{#1}}%
\newcommand{\hlcom}[1]{\textcolor[rgb]{0.678,0.584,0.686}{\textit{#1}}}%
\newcommand{\hlopt}[1]{\textcolor[rgb]{0,0,0}{#1}}%
\newcommand{\hlstd}[1]{\textcolor[rgb]{0.345,0.345,0.345}{#1}}%
\newcommand{\hlkwa}[1]{\textcolor[rgb]{0.161,0.373,0.58}{\textbf{#1}}}%
\newcommand{\hlkwb}[1]{\textcolor[rgb]{0.69,0.353,0.396}{#1}}%
\newcommand{\hlkwc}[1]{\textcolor[rgb]{0.333,0.667,0.333}{#1}}%
\newcommand{\hlkwd}[1]{\textcolor[rgb]{0.737,0.353,0.396}{\textbf{#1}}}%
\let\hlipl\hlkwb

\usepackage{framed}
\makeatletter
\newenvironment{kframe}{%
 \def\at@end@of@kframe{}%
 \ifinner\ifhmode%
  \def\at@end@of@kframe{\end{minipage}}%
  \begin{minipage}{\columnwidth}%
 \fi\fi%
 \def\FrameCommand##1{\hskip\@totalleftmargin \hskip-\fboxsep
 \colorbox{shadecolor}{##1}\hskip-\fboxsep
     % There is no \\@totalrightmargin, so:
     \hskip-\linewidth \hskip-\@totalleftmargin \hskip\columnwidth}%
 \MakeFramed {\advance\hsize-\width
   \@totalleftmargin\z@ \linewidth\hsize
   \@setminipage}}%
 {\par\unskip\endMakeFramed%
 \at@end@of@kframe}
\makeatother

\definecolor{shadecolor}{rgb}{.97, .97, .97}
\definecolor{messagecolor}{rgb}{0, 0, 0}
\definecolor{warningcolor}{rgb}{1, 0, 1}
\definecolor{errorcolor}{rgb}{1, 0, 0}
\newenvironment{knitrout}{}{} % an empty environment to be redefined in TeX

\usepackage{alltt}   % Print solutions
\usepackage{epsfig}
\usepackage{graphicx}
\usepackage{color}
\usepackage{amsmath}
\usepackage{amssymb}
\usepackage{amsthm}
\usepackage{lscape}
\usepackage{setspace}
\usepackage{hyperref}
\usepackage{multicol}
\IfFileExists{upquote.sty}{\usepackage{upquote}}{}
\begin{document}

\singlespacing
%\onehalfspacing
%\doublespacing


\title{Bayesian Computation Ch 2 Assignment}

\author{}
\date{\today}

\maketitle





% \noindent 

\begin{questions}
\question Simulate a sample from a posterior distribution when you have a histogram prior.
\begin{solution}
Let's start by defining a function to generate the values for the histogram prior:
\begin{knitrout}\footnotesize
\definecolor{shadecolor}{rgb}{0.969, 0.969, 0.969}\color{fgcolor}\begin{kframe}
\begin{alltt}
\hlcom{#Function to generate prior values for each value in x}
\hlstd{get_histprior_value} \hlkwb{<-} \hlkwa{function}\hlstd{(}\hlkwc{x}\hlstd{,} \hlkwc{histprior}\hlstd{)\{}
  \hlstd{vec} \hlkwb{<-} \hlkwd{rep}\hlstd{(}\hlnum{NA}\hlstd{,} \hlkwc{length}\hlstd{=}\hlkwd{length}\hlstd{(x))}
  \hlkwa{for}\hlstd{(i} \hlkwa{in} \hlnum{1}\hlopt{:}\hlkwd{length}\hlstd{(vec))\{}
    \hlstd{new_vec} \hlkwb{<-} \hlstd{histprior[histprior[,}\hlnum{1}\hlstd{]}\hlopt{>=}\hlstd{x[i],]}

    \hlcom{#Check if new_vec is matrix or vector and index accordingly}
    \hlkwa{if}\hlstd{(}\hlopt{!}\hlkwd{is.null}\hlstd{(}\hlkwd{ncol}\hlstd{(new_vec)))\{}
      \hlstd{vec[i]} \hlkwb{<-} \hlstd{new_vec[}\hlnum{1}\hlstd{,}\hlnum{2}\hlstd{]}
    \hlstd{\}} \hlkwa{else}\hlstd{\{}
      \hlstd{vec[i]} \hlkwb{<-} \hlstd{new_vec[}\hlnum{2}\hlstd{]}
    \hlstd{\}}
  \hlstd{\}}

  \hlkwd{return}\hlstd{(vec)}
\hlstd{\}}
\end{alltt}
\end{kframe}
\end{knitrout}
\newpage
Now let's prepare the data for input into our new \texttt{get\_histprior\_value} function:
\begin{knitrout}\footnotesize
\definecolor{shadecolor}{rgb}{0.969, 0.969, 0.969}\color{fgcolor}\begin{kframe}
\begin{alltt}
\hlcom{#Generate points for interval}
\hlstd{interval} \hlkwb{<-} \hlkwd{seq}\hlstd{(}\hlnum{0.1}\hlstd{,} \hlnum{1}\hlstd{,} \hlkwc{by} \hlstd{=} \hlnum{0.1}\hlstd{)}

\hlcom{#Prior probability}
\hlstd{prior} \hlkwb{<-} \hlkwd{c}\hlstd{(}\hlnum{1}\hlstd{,} \hlnum{5.2}\hlstd{,} \hlnum{8}\hlstd{,} \hlnum{7.2}\hlstd{,} \hlnum{4.6}\hlstd{,} \hlnum{2.1}\hlstd{,} \hlnum{0.7}\hlstd{,} \hlnum{0.1}\hlstd{,} \hlnum{0}\hlstd{,} \hlnum{0}\hlstd{)}
\hlstd{prior} \hlkwb{<-} \hlstd{prior}\hlopt{/}\hlkwd{sum}\hlstd{(prior)}

\hlcom{#Create the histogram prior}
\hlstd{histprior} \hlkwb{<-} \hlkwd{sample}\hlstd{(interval,} \hlnum{10000}\hlstd{,} \hlkwc{replace}\hlstd{=}\hlnum{TRUE}\hlstd{,} \hlkwc{prob} \hlstd{= prior)}
\hlstd{histprior} \hlkwb{<-} \hlkwd{table}\hlstd{(histprior)}\hlopt{/}\hlkwd{sum}\hlstd{(}\hlkwd{table}\hlstd{(histprior))}
\hlstd{histprior} \hlkwb{<-} \hlkwd{as.matrix}\hlstd{(histprior)}
\hlstd{names} \hlkwb{<-} \hlkwd{rownames}\hlstd{(histprior)}
\hlkwd{rownames}\hlstd{(histprior)} \hlkwb{<-} \hlkwa{NULL}
\hlstd{histprior} \hlkwb{<-} \hlkwd{cbind}\hlstd{(}\hlkwd{as.numeric}\hlstd{(names),}\hlkwd{as.numeric}\hlstd{(histprior))}
\hlstd{histprior} \hlkwb{<-} \hlkwd{rbind}\hlstd{(histprior,} \hlkwd{c}\hlstd{(}\hlnum{0.9}\hlstd{,} \hlnum{0}\hlstd{))}
\hlstd{histprior} \hlkwb{<-} \hlkwd{rbind}\hlstd{(histprior,} \hlkwd{c}\hlstd{(}\hlnum{1.0}\hlstd{,} \hlnum{0}\hlstd{))}
\end{alltt}
\end{kframe}
\end{knitrout}
\newpage
Now let's plot the prior:
\begin{knitrout}\footnotesize
\definecolor{shadecolor}{rgb}{0.969, 0.969, 0.969}\color{fgcolor}\begin{kframe}
\begin{alltt}
\hlcom{#Output histogram of prior}
\hlkwd{curve}\hlstd{(}\hlkwd{get_histprior_value}\hlstd{(x, histprior),} \hlkwc{from}\hlstd{=}\hlnum{0}\hlstd{,} \hlkwc{to} \hlstd{=} \hlnum{1}\hlstd{,}
      \hlkwc{ylim} \hlstd{=} \hlkwd{c}\hlstd{(}\hlnum{0}\hlstd{,} \hlnum{0.3}\hlstd{),} \hlkwc{n} \hlstd{=} \hlnum{10000}\hlstd{,}
      \hlkwc{xlab}\hlstd{=}\hlstr{"p"}\hlstd{,} \hlkwc{ylab} \hlstd{=} \hlstr{"Prior density"}\hlstd{,} \hlkwc{main} \hlstd{=} \hlstr{"Histogram Prior"}\hlstd{)}
\end{alltt}
\end{kframe}
\includegraphics[width=\maxwidth]{figure/unnamed-chunk-4-1} 

\end{knitrout}
\newpage
Finally, let's generate and plot the posterior:
\begin{knitrout}\footnotesize
\definecolor{shadecolor}{rgb}{0.969, 0.969, 0.969}\color{fgcolor}\begin{kframe}
\begin{alltt}
\hlstd{s} \hlkwb{<-} \hlnum{11}
\hlstd{f} \hlkwb{<-} \hlnum{16}

\hlstd{p} \hlkwb{<-} \hlkwd{seq}\hlstd{(}\hlnum{0}\hlstd{,} \hlnum{1}\hlstd{,} \hlkwc{length} \hlstd{=} \hlnum{10000}\hlstd{)}

\hlstd{post} \hlkwb{<-} \hlkwd{get_histprior_value}\hlstd{(p, histprior)} \hlopt{*} \hlkwd{dbeta}\hlstd{(p, s}\hlopt{+}\hlnum{1}\hlstd{, f}\hlopt{+}\hlnum{1}\hlstd{)}

\hlstd{post} \hlkwb{<-} \hlstd{post}\hlopt{/}\hlkwd{sum}\hlstd{(post)}

\hlstd{ps} \hlkwb{<-} \hlkwd{sample}\hlstd{(p,} \hlkwc{replace} \hlstd{=} \hlnum{TRUE}\hlstd{,} \hlkwc{prob} \hlstd{= post)}

\hlkwd{hist}\hlstd{(ps,} \hlkwc{xlab}\hlstd{=}\hlstr{"p"}\hlstd{,} \hlkwc{main}\hlstd{=}\hlstr{"Simulated Draws from the Posterior Distribution of p"}\hlstd{)}
\end{alltt}
\end{kframe}
\includegraphics[width=\maxwidth]{figure/unnamed-chunk-5-1} 

\end{knitrout}

\end{solution}

\end{questions}
\end{document}
