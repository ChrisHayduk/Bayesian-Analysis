%\documentclass[12pt,addpoints]{exam}   % Print w/o solutions
\documentclass[12pt,addpoints,answers]{exam}\usepackage[]{graphicx}\usepackage[]{color}
%% maxwidth is the original width if it is less than linewidth
%% otherwise use linewidth (to make sure the graphics do not exceed the margin)
\makeatletter
\def\maxwidth{ %
  \ifdim\Gin@nat@width>\linewidth
    \linewidth
  \else
    \Gin@nat@width
  \fi
}
\makeatother

\definecolor{fgcolor}{rgb}{0.345, 0.345, 0.345}
\newcommand{\hlnum}[1]{\textcolor[rgb]{0.686,0.059,0.569}{#1}}%
\newcommand{\hlstr}[1]{\textcolor[rgb]{0.192,0.494,0.8}{#1}}%
\newcommand{\hlcom}[1]{\textcolor[rgb]{0.678,0.584,0.686}{\textit{#1}}}%
\newcommand{\hlopt}[1]{\textcolor[rgb]{0,0,0}{#1}}%
\newcommand{\hlstd}[1]{\textcolor[rgb]{0.345,0.345,0.345}{#1}}%
\newcommand{\hlkwa}[1]{\textcolor[rgb]{0.161,0.373,0.58}{\textbf{#1}}}%
\newcommand{\hlkwb}[1]{\textcolor[rgb]{0.69,0.353,0.396}{#1}}%
\newcommand{\hlkwc}[1]{\textcolor[rgb]{0.333,0.667,0.333}{#1}}%
\newcommand{\hlkwd}[1]{\textcolor[rgb]{0.737,0.353,0.396}{\textbf{#1}}}%
\let\hlipl\hlkwb

\usepackage{framed}
\makeatletter
\newenvironment{kframe}{%
 \def\at@end@of@kframe{}%
 \ifinner\ifhmode%
  \def\at@end@of@kframe{\end{minipage}}%
  \begin{minipage}{\columnwidth}%
 \fi\fi%
 \def\FrameCommand##1{\hskip\@totalleftmargin \hskip-\fboxsep
 \colorbox{shadecolor}{##1}\hskip-\fboxsep
     % There is no \\@totalrightmargin, so:
     \hskip-\linewidth \hskip-\@totalleftmargin \hskip\columnwidth}%
 \MakeFramed {\advance\hsize-\width
   \@totalleftmargin\z@ \linewidth\hsize
   \@setminipage}}%
 {\par\unskip\endMakeFramed%
 \at@end@of@kframe}
\makeatother

\definecolor{shadecolor}{rgb}{.97, .97, .97}
\definecolor{messagecolor}{rgb}{0, 0, 0}
\definecolor{warningcolor}{rgb}{1, 0, 1}
\definecolor{errorcolor}{rgb}{1, 0, 0}
\newenvironment{knitrout}{}{} % an empty environment to be redefined in TeX

\usepackage{alltt}   % Print solutions
%%%% comment out ONE of the above lines  
%    - the first line prints the document without the solutions, just questions
%    - the second prints the document with solutions.
\usepackage{epsfig}
\usepackage{graphicx}
\usepackage{color}
\usepackage{amsmath}
\usepackage{amssymb}
\usepackage{amsthm}
\usepackage{lscape}
\usepackage{setspace}
\usepackage{hyperref}
\usepackage{multicol}
\IfFileExists{upquote.sty}{\usepackage{upquote}}{}
\begin{document}

\singlespacing
%\onehalfspacing
%\doublespacing


\title{Assignment Covering Chapter 4 of \emph{Bayesian Computation with R}}

\author{Chris Hayduk}
\date{\today}

\maketitle



\begin{questions}
\question \textbf{Chapter 4: Multiparameter Models} Write your own code for the following:
%%%%%%%%%
\begin{parts}
\part \texttt{normchi2post()} on pg. 64
%%%%%%%%%
\part \texttt{mycontour()} on pg. 64
%%%%%%%%%
\part \texttt{normpostsim()} on pg. 64
%%%%%%%%%
\part \texttt{rdirichlet()} on pg. 66
%%%%%%%%%
\part \texttt{beta.select()} on pg. 71
%%%%%%%%%
\part \texttt{logisticpost()} on pg. 72
%%%%%%%%%
%\part \texttt{simcontour()} on pg. 73
%%%%%%%%%
\part \texttt{howardprior()} on pg. 77
\end{parts}
%%%%%%%%%
\begin{solution}
\begin{parts}
  \part 
\begin{knitrout}\footnotesize
\definecolor{shadecolor}{rgb}{0.969, 0.969, 0.969}\color{fgcolor}\begin{kframe}
\begin{alltt}
\hlstd{normchi2post} \hlkwb{<-} \hlkwa{function}\hlstd{(}\hlkwc{theta}\hlstd{,} \hlkwc{data}\hlstd{)\{}
  \hlstd{mean} \hlkwb{<-} \hlstd{theta[}\hlnum{1}\hlstd{]}
  \hlstd{variance} \hlkwb{<-} \hlstd{theta[}\hlnum{2}\hlstd{]}

  \hlstd{y_bar} \hlkwb{<-} \hlkwd{mean}\hlstd{(data)}

  \hlstd{S} \hlkwb{<-} \hlkwd{sum}\hlstd{((data}\hlopt{-}\hlstd{y_bar)}\hlopt{^}\hlnum{2}\hlstd{)}
  \hlstd{n} \hlkwb{<-} \hlkwd{length}\hlstd{(data)}

  \hlstd{posterior_density} \hlkwb{<-} \hlnum{1}\hlopt{/}\hlstd{((variance}\hlopt{^}\hlstd{(n}\hlopt{/}\hlnum{2}\hlopt{+}\hlnum{1}\hlstd{)))} \hlopt{*}
    \hlkwd{exp}\hlstd{(}\hlopt{-}\hlnum{1}\hlopt{/}\hlstd{(}\hlnum{2}\hlopt{*}\hlstd{variance)}\hlopt{*}\hlstd{(S}\hlopt{+}\hlstd{n}\hlopt{*}\hlstd{(mean}\hlopt{-}\hlstd{y_bar)}\hlopt{^}\hlnum{2}\hlstd{))}

  \hlstd{log_density} \hlkwb{<-} \hlkwd{log}\hlstd{(posterior_density)}
  \hlcom{#print(log_density)}
  \hlkwd{return}\hlstd{(log_density)}
\hlstd{\}}

\hlcom{#Example run}

\hlkwd{data}\hlstd{(}\hlstr{"marathontimes"}\hlstd{)}
\hlkwd{attach}\hlstd{(marathontimes)}

\hlstd{d} \hlkwb{<-} \hlstd{LearnBayes}\hlopt{::}\hlkwd{mycontour}\hlstd{(normchi2post,} \hlkwd{c}\hlstd{(}\hlnum{220}\hlstd{,} \hlnum{330}\hlstd{,} \hlnum{500}\hlstd{,} \hlnum{9000}\hlstd{), time,} \hlkwc{xlab}\hlstd{=}\hlstr{"mean"}\hlstd{,}
               \hlkwc{ylab}\hlstd{=}\hlstr{"variance"}\hlstd{)}
\end{alltt}
\end{kframe}
\includegraphics[width=\maxwidth]{figure/unnamed-chunk-2-1} 

\end{knitrout}
  
  \part
\begin{knitrout}\footnotesize
\definecolor{shadecolor}{rgb}{0.969, 0.969, 0.969}\color{fgcolor}\begin{kframe}
\begin{alltt}
\hlstd{mycontour} \hlkwb{<-} \hlkwa{function} \hlstd{(}\hlkwc{logf}\hlstd{,} \hlkwc{limits}\hlstd{,} \hlkwc{data}\hlstd{,} \hlkwc{num_points}\hlstd{,} \hlkwc{levels}\hlstd{,} \hlkwc{...}\hlstd{)}
\hlstd{\{}
    \hlstd{LOGF} \hlkwb{=} \hlkwa{function}\hlstd{(}\hlkwc{theta}\hlstd{,} \hlkwc{data}\hlstd{) \{}
        \hlkwa{if} \hlstd{(}\hlkwd{is.matrix}\hlstd{(theta)} \hlopt{==} \hlnum{TRUE}\hlstd{) \{}
            \hlstd{val} \hlkwb{=} \hlkwd{matrix}\hlstd{(}\hlnum{0}\hlstd{,} \hlkwd{c}\hlstd{(}\hlkwd{dim}\hlstd{(theta)[}\hlnum{1}\hlstd{],} \hlnum{1}\hlstd{))}
            \hlkwa{for} \hlstd{(j} \hlkwa{in} \hlnum{1}\hlopt{:}\hlkwd{dim}\hlstd{(theta)[}\hlnum{1}\hlstd{]) val[j]} \hlkwb{=} \hlkwd{logf}\hlstd{(theta[j,}
                \hlstd{], data)}
        \hlstd{\}}
        \hlkwa{else} \hlstd{val} \hlkwb{=} \hlkwd{logf}\hlstd{(theta, data)}
        \hlkwd{return}\hlstd{(val)}
    \hlstd{\}}
    \hlstd{ng} \hlkwb{=} \hlstd{num_points}
    \hlstd{x0} \hlkwb{=} \hlkwd{seq}\hlstd{(limits[}\hlnum{1}\hlstd{], limits[}\hlnum{2}\hlstd{],} \hlkwc{len} \hlstd{= ng)}
    \hlstd{y0} \hlkwb{=} \hlkwd{seq}\hlstd{(limits[}\hlnum{3}\hlstd{], limits[}\hlnum{4}\hlstd{],} \hlkwc{len} \hlstd{= ng)}
    \hlstd{X} \hlkwb{=} \hlkwd{outer}\hlstd{(x0,} \hlkwd{rep}\hlstd{(}\hlnum{1}\hlstd{, ng))}
    \hlstd{Y} \hlkwb{=} \hlkwd{outer}\hlstd{(}\hlkwd{rep}\hlstd{(}\hlnum{1}\hlstd{, ng), y0)}
    \hlstd{n2} \hlkwb{=} \hlstd{ng}\hlopt{^}\hlnum{2}
    \hlstd{Z} \hlkwb{=} \hlkwd{LOGF}\hlstd{(}\hlkwd{cbind}\hlstd{(X[}\hlnum{1}\hlopt{:}\hlstd{n2], Y[}\hlnum{1}\hlopt{:}\hlstd{n2]), data)}

    \hlcom{#max_z <- optimize(LOGF, }
    \hlcom{#interval = c(c(limits[1], limits[2]), }
    \hlcom{#c(limits[3], limits[4])), data, maximum = TRUE)}
    \hlstd{max_z} \hlkwb{=} \hlkwd{max}\hlstd{(Z)}

    \hlstd{Z} \hlkwb{=} \hlstd{Z} \hlopt{-} \hlstd{max_z}
    \hlstd{Z} \hlkwb{=} \hlkwd{matrix}\hlstd{(Z,} \hlkwd{c}\hlstd{(ng, ng))}

    \hlstd{min_z} \hlkwb{<-} \hlkwd{min}\hlstd{(Z)}

    \hlstd{contours} \hlkwb{<-} \hlkwd{c}\hlstd{()}

    \hlkwa{for}\hlstd{(i} \hlkwa{in} \hlnum{1}\hlopt{:}\hlkwd{length}\hlstd{(levels))\{}
      \hlstd{contours} \hlkwb{<-} \hlkwd{c}\hlstd{(contours, (levels[i]}\hlopt{*}\hlstd{min_z))}
    \hlstd{\}}

    \hlkwd{contour}\hlstd{(x0, y0, Z,} \hlkwc{levels} \hlstd{= contours,} \hlkwc{lwd} \hlstd{=} \hlnum{2}\hlstd{,}
        \hlstd{...)}
\hlstd{\}}

\hlstd{d} \hlkwb{<-} \hlkwd{mycontour}\hlstd{(normchi2post,} \hlkwd{c}\hlstd{(}\hlnum{220}\hlstd{,} \hlnum{330}\hlstd{,} \hlnum{500}\hlstd{,} \hlnum{9000}\hlstd{),}
               \hlstd{time,} \hlkwc{num_points} \hlstd{=} \hlnum{1000}\hlstd{,}
               \hlkwc{levels} \hlstd{=} \hlkwd{c}\hlstd{(}\hlnum{0.1}\hlstd{,} \hlnum{0.01}\hlstd{,} \hlnum{0.001}\hlstd{),}
               \hlkwc{xlab}\hlstd{=}\hlstr{"mean"}\hlstd{,} \hlkwc{ylab}\hlstd{=}\hlstr{"variance"}\hlstd{)}
\end{alltt}
\end{kframe}
\includegraphics[width=\maxwidth]{figure/unnamed-chunk-3-1} 

\end{knitrout}
  
  \part
\begin{knitrout}\footnotesize
\definecolor{shadecolor}{rgb}{0.969, 0.969, 0.969}\color{fgcolor}\begin{kframe}
\begin{alltt}
\hlstd{normpostsim} \hlkwb{<-} \hlkwa{function}\hlstd{(}\hlkwc{data}\hlstd{,} \hlkwc{m}\hlstd{)\{}
  \hlstd{S} \hlkwb{<-} \hlkwd{sum}\hlstd{((data}\hlopt{-}\hlkwd{mean}\hlstd{(data))}\hlopt{^}\hlnum{2}\hlstd{)}

  \hlstd{n} \hlkwb{<-} \hlkwd{length}\hlstd{(data)}

  \hlstd{sigma2} \hlkwb{<-} \hlstd{S}\hlopt{/}\hlkwd{rchisq}\hlstd{(m, n}\hlopt{-}\hlnum{1}\hlstd{)}

  \hlstd{mu} \hlkwb{<-} \hlkwd{rnorm}\hlstd{(m,} \hlkwc{mean} \hlstd{=} \hlkwd{mean}\hlstd{(data),} \hlkwc{sd} \hlstd{=} \hlkwd{sqrt}\hlstd{(sigma2)}\hlopt{/}\hlkwd{sqrt}\hlstd{(n))}

  \hlstd{results} \hlkwb{<-} \hlkwd{data.frame}\hlstd{(}\hlkwc{mu} \hlstd{= mu,} \hlkwc{sigma2} \hlstd{= sigma2)}

  \hlkwd{return}\hlstd{(results)}
\hlstd{\}}

\hlcom{#Example run}
\hlstd{results} \hlkwb{<-} \hlkwd{normpostsim}\hlstd{(time,} \hlnum{1000}\hlstd{)}
\hlstd{d} \hlkwb{<-} \hlkwd{mycontour}\hlstd{(normchi2post,} \hlkwd{c}\hlstd{(}\hlnum{220}\hlstd{,} \hlnum{330}\hlstd{,} \hlnum{500}\hlstd{,} \hlnum{9000}\hlstd{), time,}
               \hlkwc{num_points} \hlstd{=} \hlnum{1000}\hlstd{, levels} \hlkwb{<-} \hlkwd{c}\hlstd{(}\hlnum{0.1}\hlstd{,} \hlnum{0.01}\hlstd{,} \hlnum{0.001}\hlstd{),}
               \hlkwc{xlab}\hlstd{=}\hlstr{"mean"}\hlstd{,} \hlkwc{ylab}\hlstd{=}\hlstr{"variance"}\hlstd{)}
\hlkwd{points}\hlstd{(results}\hlopt{$}\hlstd{mu, results}\hlopt{$}\hlstd{sigma2)}
\end{alltt}
\end{kframe}
\includegraphics[width=\maxwidth]{figure/unnamed-chunk-4-1} 

\end{knitrout}
  
  \part
\begin{knitrout}\footnotesize
\definecolor{shadecolor}{rgb}{0.969, 0.969, 0.969}\color{fgcolor}\begin{kframe}
\begin{alltt}
\hlstd{rdirichlet} \hlkwb{<-} \hlkwa{function}\hlstd{(}\hlkwc{m}\hlstd{,} \hlkwc{alpha}\hlstd{)\{}
  \hlstd{theta} \hlkwb{<-} \hlkwd{matrix}\hlstd{(}\hlkwc{nrow} \hlstd{= m,} \hlkwc{ncol} \hlstd{=} \hlkwd{length}\hlstd{(alpha))}

  \hlkwa{for}\hlstd{(i} \hlkwa{in} \hlnum{1}\hlopt{:}\hlkwd{length}\hlstd{(alpha))\{}
    \hlstd{theta[,i]} \hlkwb{<-} \hlkwd{rgamma}\hlstd{(m, alpha[i],} \hlnum{1}\hlstd{)}
  \hlstd{\}}

  \hlkwa{for}\hlstd{(i} \hlkwa{in} \hlnum{1}\hlopt{:}\hlkwd{nrow}\hlstd{(theta))\{}
    \hlstd{T} \hlkwb{=} \hlkwd{sum}\hlstd{(theta[i,])}

    \hlstd{theta[i,]} \hlkwb{<-} \hlstd{theta[i,]}\hlopt{/}\hlstd{T}
  \hlstd{\}}

  \hlkwd{return}\hlstd{(theta)}
\hlstd{\}}

\hlstd{results} \hlkwb{<-} \hlkwd{rdirichlet}\hlstd{(}\hlnum{1000}\hlstd{,} \hlkwd{c}\hlstd{(}\hlnum{728}\hlstd{,}\hlnum{584}\hlstd{,}\hlnum{138}\hlstd{))}

\hlkwd{hist}\hlstd{(results[,}\hlnum{1}\hlstd{]} \hlopt{-} \hlstd{results[,}\hlnum{2}\hlstd{],} \hlkwc{main} \hlstd{=} \hlstr{""}\hlstd{)}
\end{alltt}
\end{kframe}
\includegraphics[width=\maxwidth]{figure/unnamed-chunk-5-1} 

\end{knitrout}
  
  \part
\begin{knitrout}\footnotesize
\definecolor{shadecolor}{rgb}{0.969, 0.969, 0.969}\color{fgcolor}\begin{kframe}
\begin{alltt}
\hlstd{beta.select} \hlkwb{<-} \hlkwa{function}\hlstd{(}\hlkwc{quantile1}\hlstd{,} \hlkwc{quantile2}\hlstd{)\{}

\hlstd{\}}

\hlcom{#Example run}
\hlkwd{beta.select}\hlstd{(}\hlkwd{list}\hlstd{(}\hlkwc{p}\hlstd{=}\hlnum{0.5}\hlstd{,} \hlkwc{x}\hlstd{=}\hlnum{0.2}\hlstd{),} \hlkwd{list}\hlstd{(}\hlkwc{p}\hlstd{=}\hlnum{0.9}\hlstd{,} \hlkwc{x}\hlstd{=}\hlnum{0.5}\hlstd{))}
\end{alltt}
\begin{verbatim}
## NULL
\end{verbatim}
\end{kframe}
\end{knitrout}
  
  \part
\begin{knitrout}\footnotesize
\definecolor{shadecolor}{rgb}{0.969, 0.969, 0.969}\color{fgcolor}\begin{kframe}
\begin{alltt}
\hlstd{logisticpost} \hlkwb{<-} \hlkwa{function}\hlstd{(}\hlkwc{beta}\hlstd{,} \hlkwc{data}\hlstd{)\{}
  \hlstd{post} \hlkwb{<-} \hlnum{1}

  \hlkwa{for}\hlstd{(i} \hlkwa{in} \hlnum{1}\hlopt{:}\hlkwd{nrow}\hlstd{(data))\{}
    \hlstd{p} \hlkwb{<-} \hlkwd{exp}\hlstd{(beta[}\hlnum{1}\hlstd{]} \hlopt{+} \hlstd{beta[}\hlnum{2}\hlstd{]}\hlopt{*}\hlstd{data[i,} \hlnum{1}\hlstd{])}

    \hlstd{p} \hlkwb{<-} \hlstd{p}\hlopt{/}\hlstd{(}\hlnum{1}\hlopt{+}\hlkwd{exp}\hlstd{(beta[}\hlnum{1}\hlstd{]} \hlopt{+} \hlstd{beta[}\hlnum{2}\hlstd{]}\hlopt{*}\hlstd{data[i,} \hlnum{1}\hlstd{]))}

    \hlstd{y} \hlkwb{<-} \hlstd{data[i,} \hlnum{3}\hlstd{]}
    \hlstd{n} \hlkwb{<-} \hlstd{data[i,} \hlnum{2}\hlstd{]}

    \hlstd{post} \hlkwb{<-} \hlstd{post} \hlopt{*} \hlstd{(p}\hlopt{^}\hlstd{y} \hlopt{*} \hlstd{(}\hlnum{1}\hlopt{-}\hlstd{p)}\hlopt{^}\hlstd{(n}\hlopt{-}\hlstd{y))}
  \hlstd{\}}

  \hlkwd{return}\hlstd{(}\hlkwd{log}\hlstd{(post))}
\hlstd{\}}

\hlcom{#Example run}
\hlstd{x} \hlkwb{<-} \hlkwd{c}\hlstd{(}\hlopt{-}\hlnum{0.86}\hlstd{,} \hlopt{-}\hlnum{0.3}\hlstd{,} \hlopt{-}\hlnum{0.05}\hlstd{,} \hlnum{0.73}\hlstd{)}
\hlstd{n} \hlkwb{<-} \hlkwd{c}\hlstd{(}\hlnum{5}\hlstd{,} \hlnum{5}\hlstd{,} \hlnum{5}\hlstd{,} \hlnum{5}\hlstd{)}
\hlstd{y} \hlkwb{<-} \hlkwd{c}\hlstd{(}\hlnum{0}\hlstd{,} \hlnum{1}\hlstd{,} \hlnum{3}\hlstd{,} \hlnum{5}\hlstd{)}
\hlstd{data} \hlkwb{<-} \hlkwd{cbind}\hlstd{(x, n, y)}

\hlstd{prior} \hlkwb{<-} \hlkwd{rbind}\hlstd{(}\hlkwd{c}\hlstd{(}\hlopt{-}\hlnum{0.7}\hlstd{,} \hlnum{4.68}\hlstd{,} \hlnum{1.12}\hlstd{),}
               \hlkwd{c}\hlstd{(}\hlnum{0.6}\hlstd{,} \hlnum{2.10}\hlstd{,} \hlnum{0.74}\hlstd{))}

\hlstd{data.new} \hlkwb{<-} \hlkwd{rbind}\hlstd{(data,prior)}

\hlkwd{mycontour}\hlstd{(logisticpost,} \hlkwd{c}\hlstd{(}\hlopt{-}\hlnum{3}\hlstd{,}\hlnum{3}\hlstd{,}\hlopt{-}\hlnum{1}\hlstd{,}\hlnum{9}\hlstd{), data.new,}
          \hlkwc{num_points} \hlstd{=} \hlnum{1000}\hlstd{, levels} \hlkwb{<-} \hlkwd{c}\hlstd{(}\hlnum{0.1}\hlstd{,} \hlnum{0.01}\hlstd{,} \hlnum{0.001}\hlstd{),}
          \hlkwc{xlab}\hlstd{=}\hlstr{"beta0"}\hlstd{,} \hlkwc{ylab}\hlstd{=}\hlstr{"beta1"}\hlstd{)}
\end{alltt}
\end{kframe}
\includegraphics[width=\maxwidth]{figure/unnamed-chunk-7-1} 

\end{knitrout}
  
  \part
\begin{knitrout}\footnotesize
\definecolor{shadecolor}{rgb}{0.969, 0.969, 0.969}\color{fgcolor}\begin{kframe}
\begin{alltt}
\hlstd{howardprior} \hlkwb{<-} \hlkwa{function}\hlstd{()\{}

\hlstd{\}}
\end{alltt}
\end{kframe}
\end{knitrout}
  
\end{parts}
\end{solution}
%%%%%%%%%%%%%%%%%%%%%
\question \textbf{Chapter 4: Multiparameter Models} Pg. 66 introduces the Dirichlet distribution.  Draw pdfs of this distribution for various parameter values using the functions in the R package. Find some examples on the internet of where the Dirichlet distribution is used in statistics.  
%%%%%%%%%
\begin{solution}
\begin{knitrout}\footnotesize
\definecolor{shadecolor}{rgb}{0.969, 0.969, 0.969}\color{fgcolor}
\includegraphics[width=\maxwidth]{figure/unnamed-chunk-9-1} 

\end{knitrout}

\end{solution}
%%%%%%%%%%%%%%%%%%%%%
\question Do all CH. 4 exercises.
%%%%%%%%%
\begin{solution}

\end{solution}
%%%%%%%%%%%%%%%%%%%%%
\question \textbf{CH4, Q8 extra:} fit model using a frequentist logistic regression model; figure out how to make the equivalent confidence interval for part (d).  (Could use bootstrap, probably BC\_a and its approximation ABC, try package \texttt{\emph{bootBCa}}; include a histogram of your BC\_a bootstrap resampled probability values and normal quantile plot of those values.  Let number of bootstrap samples be $B=9,999$.)
\end{questions}
%%%%%%%%%
\begin{solution}

\end{solution}

\end{document}
